\documentclass{book} 
% ====================== 
% METADATOS DEL LIBRO 
% ====================== 
\title{Bioquímica I} 
\newcommand{\booksubtitle}{Fundamentos Moleculares de la Vida} 
\newcommand{\booklicense}{Creative Commons Zero 1.0 Universal} 
\author{Aldo F. Combariza \\ Facultad de Educación y Ciencias \\ Programa de Biología} 
\newcommand{\authorsubtitle}{Universidad de Sucre, Colombia} 

% Create convenient commands \booktitle and \bookauthor 
\makeatletter 
\newcommand{\booktitle}{\@title} 
\newcommand{\bookauthor}{\@author} 
\makeatother 

% ====================== 
% CONFIGURACIÓN DE PAQUETES 
% ====================== 
\usepackage[utf8]{inputenc} 
\usepackage[spanish]{babel} 
\usepackage{amssymb} 
\usepackage{fix-cm} 
\usepackage{tikz} 
\usepackage{amsmath} 
\usepackage{chemformula} 
\usepackage{graphicx} 
\usepackage{booktabs} 
\usepackage{textgreek} 
\usepackage{makeidx} 
\usepackage[table]{xcolor} 
\usepackage{makeidx} 
\usepackage{newunicodechar} 
\newunicodechar{↔}{\leftrightarrow} 
\makeindex 
\usepackage{tabularx} 

% Configuración de dimensiones para libro impreso 
\usepackage[bindingoffset=0.625in, left=.5in, right=.5in, top=.8125in, bottom=.9375in, paperwidth=6.375in, paperheight=9.25in]{geometry} 
\usepackage{tikz} 
\usetikzlibrary{arrows.meta} 
\tikzset{arrow/.style={-{Stealth[scale=1.2]}, thick}} 
\renewcommand{\contentsname}{Índice General} 

% ====================== 
% CONTENIDO PRINCIPAL 
% ====================== 
\begin{document} 
\frontmatter 
% ---- PÁGINA DE MEDIO TÍTULO ---- 
\newgeometry{top=1.75in,bottom=.5in} 
\begin{titlepage} 
\begin{flushleft} 
\textbf{\fontfamily{qcs}\fontsize{48}{54}\selectfont Bioquímica\\I\\} 
\par\noindent\rule{\textwidth}{4pt}\\ 
\begin{tikzpicture} 
\shade[bottom color=lightgray,top color=white] (0,0) rectangle (\textwidth, 1.5) node[midway] {\textbf{\large \textit{\booksubtitle}}}; 
\end{tikzpicture} 
\begin{flushright} \Large Primera Edición \end{flushright} 
vspace{\fill} 
\end{flushleft} 
\end{titlepage} 
\restoregeometry 
\thispagestyle{empty} 
% ---- PÁGINA DE TÍTULO PRINCIPAL ---- 
\newgeometry{top=1.75in,bottom=.5in} 
\begin{titlepage} 
\begin{flushleft} 
\textbf{\fontfamily{qcs}\fontsize{48}{54}\selectfont Bioquímica\\I\\} 
\par\noindent\rule{\textwidth}{4pt}\\ 
\begin{tikzpicture} 
\shade[bottom color=lightgray,top color=white] (0,0) rectangle (\textwidth, 1.5) node[midway] {\textbf{\large \textit{\booksubtitle}}}; 
\end{tikzpicture} 
\begin{flushright} \Large Primera Edición \end{flushright} 
vspace{\fill} 
\textbf{\large \bookauthor}\\[3.5pt] 
\textbf{\large \textit{\authorsubtitle}} 
vspace{\fill} 
\begin{center} 
% Descomentar cuando tengas logo 
\includegraphics[width=0.3\textwidth]{logo.pdf} 
% \\[4pt] 
\fontfamily{lmtt}\small{Publicación Académica\\ Facultad de Educación y Ciencias\\ Universidad de Sucre} 
\end{center} 
\end{flushleft} 
\end{titlepage} 
\restoregeometry 
\thispagestyle{empty} 
% ---- COLOFÓN ---- 
\begin{flushleft} 
vspace*{\fill} 
Este libro de texto fue desarrollado para la asignatura Bioquímica I (Código 101283) del Programa de Biología.\\ Tipografiado usando \LaTeX.\\ 
vspace{\fill} 
\textcopyright{} \the\year{}  \bookauthor\\ Licencia: \booklicense 
\end{flushleft} 
\addtocounter{page}{2} 
% ---- PREFACIO ---- 
\chapter*{Prefacio} 
Este libro \textit{Bioquímica I: Fundamentos Moleculares de la Vida} ha sido desarrollado como material de apoyo central para la asignatura Bioquímica I del Programa de Biología de la Universidad de Sucre. La Bioquímica constituye uno de los pilares fundamentales para la comprensión de los seres vivos. Esta disciplina nos permite sentar las bases para entender los fenómenos que ocurren en los organismos y su papel en los procesos metabólicos, siendo una de las áreas que mayor desarrollo ha alcanzado en los últimos siglos. El presente texto se estructura en cuatro partes que guiarán al estudiante desde los conceptos fundamentales hasta las aplicaciones prácticas: 
\begin{itemize} 
\item \textbf{Parte I: Fundamentos Conceptuales} - Introduce los principios básicos de la química de la vida 
\item \textbf{Parte II: Biomoléculas: Estructura y Función} - Describe detalladamente cada grupo de biomoléculas 
\item \textbf{Parte III: Compuestos Auxiliares y Metabolismo Básico} - Aborda vitaminas, minerales y principios metabólicos 
\item \textbf{Parte IV: Aplicaciones y Contexto} - Relaciona la bioquímica con salud, ambiente y técnicas de laboratorio 
\end{itemize} 
Cada capítulo incluye objetivos de aprendizaje, ejemplos contextualizados, conceptos clave y referencias a investigaciones relevantes para nuestra región, particularmente del Caribe colombiano. Esperamos que este material sea una herramienta valiosa en la formación de los futuros biólogos y contribuya al fortalecimiento de la educación científica en nuestra institución. 
vspace{1cm} 
\begin{flushright} 
\textit{Los autores}\\ 
\textit{Sucre, Colombia}\\ 
\textit{\today} 
\end{flushright} 
% ---- ÍNDICE GENERAL ---- 
\setcounter{tocdepth}{3} 
\tableofcontents 
\mainmatter 
% =========================== 
% PARTE I: FUNDAMENTOS 
% =========================== 
\part{Fundamentos Conceptuales} 
# Chapter 0: Prerequisite Material and Digital Tools for Biochemistry

## 1. Introduction
In this chapter, we will cover the necessary prerequisite material and the digital tools that students will need to effectively navigate the field of Biochemistry.

## 2. Prerequisite Material
Before delving into Biochemistry, it's important for students to have a good understanding of the following subjects:

- **General Chemistry**: A solid grasp of basic chemistry principles is essential, including knowledge of chemical bonds, reactions, and stoichiometry.
- **Organic Chemistry**: Familiarity with organic structures and functional groups is crucial as many biochemical reactions involve organic molecules.
- **Biology**: Basic concepts of cell biology, genetics, and microbiology will provide a foundation for understanding biochemical processes.

## 3. Digital Tools
Digital tools are increasingly important in the study of Biochemistry. Here are some recommended tools:

- **Molecular Visualization Software**: Tools such as PyMOL or Chimera allow students to visualize molecular structures and understand their functions.
- **Data Analysis Software**: Programs like R or Python's Biopython library can assist in performing complex data analyses and simulations.
- **Online Resources**: Websites like PubMed or the Protein Data Bank (PDB) provide access to a wealth of research articles and protein structures relevant to Biochemistry.

## 4. Conclusion
Equipping oneself with the right knowledge and tools is key to mastering the complexities of Biochemistry. This chapter aims to prepare students for what's to come in the following chapters of this textbook. 
\chapter{Chapter 1: Bioelements and Water}

\section{Objectives}
\begin{itemize}
    \item Understand the fundamental role of bioelements in living organisms.
    \item Analyze the physical and chemical properties of water.
    \item Learn how bioelements interact within biochemical processes.
\end{itemize}

\section{Theoretical Content}
Bioelements are the basic chemical elements that make up living organisms. These include...

\subsection{1.1 Definition of Bioelements}
Bioelements can be classified into...

\subsection{1.2 Importance of Water}
Water is essential for...

\section{Exercises}
\begin{enumerate}
    \item List the main bioelements and their functions.
    \item Discuss the unique properties of water that make it suitable for life.
\end{enumerate}

\section{Laboratory Practice}
In this section, students will engage in various practical exercises such as...

\section{Computational Activities}
Students will use computational tools to model...

\section{Self-Evaluation}
At the end of this chapter, students should be able to answer the following questions:
\begin{itemize}
    \item What are the main bioelements?
    \item How does the structure of water contribute to its properties?
\end{itemize} 
# Chapter 2: Carbohydrates

In this chapter, we will explore carbohydrates, their structure, classification, and role in biological systems. Carbohydrates are organic molecules made up of carbon, hydrogen, and oxygen, usually following the formula (CH2O)n, where 'n' is the number of carbon atoms.

## Structure of Carbohydrates

Carbohydrates can be classified into three main categories: monosaccharides, disaccharides, and polysaccharides. 

1. **Monosaccharides** - The simplest form of carbohydrates, consisting of single sugar molecules such as glucose and fructose.
2. **Disaccharides** - Formed by the combination of two monosaccharides. Common examples include sucrose (glucose + fructose) and lactose (glucose + galactose).
3. **Polysaccharides** - Long chains of monosaccharide units. Examples include starch, glycogen, and cellulose.

## Role of Carbohydrates

Carbohydrates play vital roles in various biological processes. They are a primary energy source, provide structural components in cells, and participate in cell signaling processes. 

Through this chapter, we will delve deeper into each type of carbohydrate, their functions, and their importance in nutrition and metabolism. 
% =========================== 
% BIBLIOGRAFÍA E ÍNDICE 
% =========================== 
\begin{thebibliography}{99} 
\bibitem{alberts} Alberts, B. et al. (2011). \textit{Introducción a la Biología Celular}. Editorial Médica Panamericana. 
\bibitem{nelson} Nelson, D. L., \& Cox, M. M. (2014). \textit{Principios de Bioquímica. Lehninger}. Omega. 
\bibitem{berg} Berg, J. M., Tymoczko, J. L., \& Stryer, L. (2015). \textit{Biochemistry}. W.H. Freeman. 
\bibitem{murray} Murray, R. et al. (2010). \textit{Harper Bioquímica ilustrada}. McGraw-Hill. 
\end{thebibliography} 
\backmatter 
\addcontentsline{toc}{part}{Índice Alfabético} 
\printindex 
\end{document}