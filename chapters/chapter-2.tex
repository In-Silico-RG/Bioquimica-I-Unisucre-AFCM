# Chapter 2: Carbohydrates

In this chapter, we will explore carbohydrates, their structure, classification, and role in biological systems. Carbohydrates are organic molecules made up of carbon, hydrogen, and oxygen, usually following the formula (CH2O)n, where 'n' is the number of carbon atoms.

## Structure of Carbohydrates

Carbohydrates can be classified into three main categories: monosaccharides, disaccharides, and polysaccharides. 

1. **Monosaccharides** - The simplest form of carbohydrates, consisting of single sugar molecules such as glucose and fructose.
2. **Disaccharides** - Formed by the combination of two monosaccharides. Common examples include sucrose (glucose + fructose) and lactose (glucose + galactose).
3. **Polysaccharides** - Long chains of monosaccharide units. Examples include starch, glycogen, and cellulose.

## Role of Carbohydrates

Carbohydrates play vital roles in various biological processes. They are a primary energy source, provide structural components in cells, and participate in cell signaling processes. 

Through this chapter, we will delve deeper into each type of carbohydrate, their functions, and their importance in nutrition and metabolism.